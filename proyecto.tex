\documentclass[11pt,]{article}
\usepackage[left=1in,top=1in,right=1in,bottom=1in]{geometry}
\newcommand*{\authorfont}{\fontfamily{phv}\selectfont}
\usepackage[]{mathpazo}
\usepackage{url}


  \usepackage[T1]{fontenc}
  \usepackage[utf8]{inputenc}



\usepackage{abstract}
\renewcommand{\abstractname}{}    % clear the title
\renewcommand{\absnamepos}{empty} % originally center

\renewenvironment{abstract}
 {{%
    \setlength{\leftmargin}{0mm}
    \setlength{\rightmargin}{\leftmargin}%
  }%
  \relax}
 {\endlist}

\makeatletter
\def\@maketitle{%
  \newpage
%  \null
%  \vskip 2em%
%  \begin{center}%
  \let \footnote \thanks
    {\fontsize{18}{20}\selectfont\raggedright  \setlength{\parindent}{0pt} \@title \par}%
}
%\fi
\makeatother




\setcounter{secnumdepth}{3}



\title{Distribución espacial de los puntos de calor - Hot Spots para los
incendios forestales en la República Dominicana y predicción de las
zonas mas vulnerables usando el metodo de Kriging universal  }



\author{\Large Diego Cordero\vspace{0.05in} \newline\normalsize\emph{Estudiante de Maestría en Teledetección y Ciencias de la Información
Geográfica, Universidad Autónoma de Santo Domingo (UASD)}  }


\date{}

\usepackage{titlesec}

\titleformat*{\section}{\normalsize\bfseries}
\titleformat*{\subsection}{\normalsize\itshape}
\titleformat*{\subsubsection}{\normalsize\itshape}
\titleformat*{\paragraph}{\normalsize\itshape}
\titleformat*{\subparagraph}{\normalsize\itshape}

\titlespacing{\section}
{0pt}{36pt}{0pt}
\titlespacing{\subsection}
{0pt}{36pt}{0pt}
\titlespacing{\subsubsection}
{0pt}{36pt}{0pt}





\newtheorem{hypothesis}{Hypothesis}
\usepackage{setspace}

\makeatletter
\@ifpackageloaded{hyperref}{}{%
\ifxetex
  \PassOptionsToPackage{hyphens}{url}\usepackage[setpagesize=false, % page size defined by xetex
              unicode=false, % unicode breaks when used with xetex
              xetex]{hyperref}
\else
  \PassOptionsToPackage{hyphens}{url}\usepackage[unicode=true]{hyperref}
\fi
}

\@ifpackageloaded{color}{
    \PassOptionsToPackage{usenames,dvipsnames}{color}
}{%
    \usepackage[usenames,dvipsnames]{color}
}
\makeatother
\hypersetup{breaklinks=true,
            bookmarks=true,
            pdfauthor={Diego Cordero (Estudiante de Maestría en Teledetección y Ciencias de la Información
Geográfica, Universidad Autónoma de Santo Domingo (UASD))},
             pdfkeywords = {r, rstudio, fire forest, incendio forestal, kriging, LISA cluster},  
            pdftitle={Distribución espacial de los puntos de calor - Hot Spots para los
incendios forestales en la República Dominicana y predicción de las
zonas mas vulnerables usando el metodo de Kriging universal},
            colorlinks=true,
            citecolor=blue,
            urlcolor=blue,
            linkcolor=magenta,
            pdfborder={0 0 0}}
\urlstyle{same}  % don't use monospace font for urls

% set default figure placement to htbp
\makeatletter
\def\fps@figure{htbp}
\makeatother

\usepackage{pdflscape} \newcommand{\blandscape}{\begin{landscape}}
\newcommand{\elandscape}{\end{landscape}}


% add tightlist ----------
\providecommand{\tightlist}{%
\setlength{\itemsep}{0pt}\setlength{\parskip}{0pt}}

\begin{document}
	
% \pagenumbering{arabic}% resets `page` counter to 1 
%
% \maketitle

{% \usefont{T1}{pnc}{m}{n}
\setlength{\parindent}{0pt}
\thispagestyle{plain}
{\fontsize{18}{20}\selectfont\raggedright 
\maketitle  % title \par  

}

{
   \vskip 13.5pt\relax \normalsize\fontsize{11}{12} 
\textbf{\authorfont Diego Cordero} \hskip 15pt \emph{\small Estudiante de Maestría en Teledetección y Ciencias de la Información
Geográfica, Universidad Autónoma de Santo Domingo (UASD)}   

}

}








\begin{abstract}

    \hbox{\vrule height .2pt width 39.14pc}

    \vskip 8.5pt % \small 

\noindent Esta investigación se basa en información geográfica de libre acceso
encontrada en la web y pretende realizar un análisis geoespacial que
permita conocer cuál es la distribución espacial de los puntos de calor
- Hot Spots para los incendios forestales en la República Dominicana.
Para lo cual se descargó la información correspondiente para el periodo
2009-2019 de la pagina de NASA-FIRMS (2020). El resultado final esperado
que se busca mediante el desarrollo de un script bajo el software R y la
herramienta RStudio es un Cluster LISA que indique la relación espacial
de los municipios con los incendios y una predicción de las zonas mas
vulnerables usando el metodo de Kriging universal teniendo como base la
elevación de un modelo digital de elevación (DEM).


\vskip 8.5pt \noindent \emph{Keywords}: r, rstudio, fire forest, incendio forestal, kriging, LISA cluster \par

    \hbox{\vrule height .2pt width 39.14pc}



\end{abstract}


\vskip 6.5pt


\noindent  \section{Introducción}\label{introducciuxf3n}

Principalmente en la descargada desde NASA (2020)

\section{Metodología}\label{metodologuxeda}

\subsection{Librerias Necesarias y adición de
data}\label{librerias-necesarias-y-adiciuxf3n-de-data}

Inicialmente se usarán las siguientes librerías y el archivo para el
desarrollo de los LISA cluster realizado por Martínez Batlle (2019) para
el desarrollo del script para la investigación

\hypertarget{refs}{}
\hypertarget{ref-profesorLisa}{}
Martínez Batlle, J. R. (2019). \emph{Script adición lisa cluster}.

\hypertarget{ref-firmsdw}{}
NASA. (2020). \emph{Pagina de descargas -- fire information for resource
management system (firms)}. Retrieved from
\url{https://firms.modaps.eosdis.nasa.gov/download/create.php}

\hypertarget{ref-firmsweb}{}
NASA-FIRMS. (2020). \emph{Fire information for resource management
system (firms)}. Retrieved from
\url{https://firms.modaps.eosdis.nasa.gov/}




\newpage
\singlespacing 
\end{document}
